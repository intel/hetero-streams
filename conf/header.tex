\documentclass[a4,oneside]{book}
\usepackage[margin=3cm]{geometry}
\usepackage{makeidx}
\usepackage{graphicx}
\usepackage{multicol}
\usepackage{float}
\usepackage{listings}
\usepackage{color}
\usepackage{xcolor}
\usepackage{textcomp}
\usepackage{alltt}
\usepackage{times}
\usepackage{ifpdf}
\usepackage{tabularx}
\ifpdf
\usepackage[pdftex,
            pagebackref=true,
            colorlinks=true,
            linkcolor=blue,
            unicode
           ]{hyperref}
\else
\usepackage[ps2pdf,
            pagebackref=true,
            colorlinks=true,
            linkcolor=blue,
            unicode
           ]{hyperref}
\usepackage{pspicture}
\fi
\usepackage[utf8]{inputenc}
\usepackage{doxygen}
\definecolor{periwinkle}{RGB}{121,119,184}
\definecolor{forestgreen}{RGB}{0,155,85}
\definecolor{listing@base03}{HTML}{002B36}
\definecolor{listing@base02}{HTML}{073642}
\definecolor{listing@base01}{HTML}{586e75}
\definecolor{listing@base00}{HTML}{657b83}
\definecolor{listing@base0}{HTML}{839496}
\definecolor{listing@base1}{HTML}{93a1a1}
\definecolor{listing@base2}{HTML}{EEE8D5}
\definecolor{listing@base3}{HTML}{FDF6E3}
\definecolor{listing@yellow}{HTML}{B58900}
\definecolor{listing@orange}{HTML}{CB4B16}
\definecolor{listing@red}{HTML}{DC322F}
\definecolor{listing@magenta}{HTML}{D33682}
\definecolor{listing@violet}{HTML}{6C71C4}
\definecolor{listing@blue}{HTML}{268BD2}
\definecolor{listing@cyan}{HTML}{2AA198}
\definecolor{listing@green}{HTML}{859900}
\definecolor{listing@darkgrey}{HTML}{0C090A}
\lstdefinestyle{DefaultListingStyle}{ %
  backgroundcolor=\color{white},
  basicstyle=\footnotesize\ttfamily,
  breakatwhitespace=false,
  breaklines=true,
  captionpos=b,
  %deletekeywords={...},
  extendedchars=true,
  frame=single,
  keepspaces=true,
  numbers=left,
  numbersep=5pt,
  numberstyle=\tiny\color{gray},
  rulecolor=\color{black},
  showspaces=false,
  showstringspaces=false,
  showtabs=false,
  tabsize=2,
  title=\lstname,
  rulesepcolor=\color{listing@base03},
  numberstyle=\tiny\color{listing@base01},
  basicstyle=\footnotesize\ttfamily,
  keywordstyle=\color{listing@green},
  stringstyle=\color{listing@cyan}\ttfamily,
  identifierstyle=\color{listing@darkgrey},
  commentstyle=\color{listing@base01},
  emphstyle=\color{listing@red}
}
\lstdefinestyle{CppListingStyle}{ %
  style=DefaultListingStyle,
  language=C++,
  otherkeywords={uint64\_t}
}
\lstdefinestyle{BashListingStyle}{ %
  style=DefaultListingStyle,
  language=bash
}
\lstdefinestyle{BatchListingStyle}{ %
  style=DefaultListingStyle,
  language=bash, % for lack of better
  morecomment=[l]{rem}
}
\lstdefinestyle{BashCmdListingStyle}{ %
  style=BashListingStyle,
  language=bash,
  deletekeywords={for,function}
}

\renewcommand\familydefault{\sfdefault}
\makeindex
\setcounter{tocdepth}{3}
\renewcommand{\footrulewidth}{0.4pt}
\def\arraystretch{1.3}
% get all the trademarks and so on right
\newcommand{\ixp}{Intel\textregistered{} Xeon Phi\texttrademark{}}
\newcommand{\kncixp}{Intel\textregistered{} Xeon Phi\texttrademark{} x100 family}
\newcommand{\coi}{Intel\textregistered{} Coprocessor Offload Infrastructure}
\newcommand{\mpss}{Intel\textregistered{} MPSS}
\newcommand{\mpssFullName}{Intel\textregistered{} Manycore Platform Software Stack}
\newcommand{\hstreams}{hStreams}
\newcommand{\heterostreams}{Hetero Streams Library}
\newcommand{\windows}{Windows}
\newcommand{\linux}{Linux}
\newcommand{\imgdir}{../../../doc/images/}
\begin{document}
\hypersetup{pageanchor=false}
\begin{titlepage}
\vspace*{7cm}
\begin{center}
    {\Large \heterostreams\ 1.3 Programming Guide and API Reference}\\[2em]
\end{center}
\end{titlepage}
\clearemptydoublepage
\chapter*{Legal Disclaimer}
You may not use or facilitate the use of this document in connection with any infringement or other legal analysis concerning Intel products described herein. You agree to grant Intel a non-exclusive, royalty-free license to any patent claim thereafter drafted which includes subject matter disclosed herein.

No license (express or implied, by estoppel or otherwise) to any intellectual property rights is granted by this document.

All information provided here is subject to change without notice. Contact your Intel representative to obtain the latest Intel product specifications and roadmaps.

The products described may contain design defects or errors known as errata which may cause the product to deviate from published specifications. Current characterized errata are available on request.

Copies of documents which have an order number and are referenced in this document may be obtained by calling 1-800-548-4725 or by visiting: http://www.intel.com/design/literature.htm

Intel and the Intel logo, are trademarks of Intel Corporation in the U.S. and/or other countries.

*Other names and brands may be claimed as the property of others.

Copyright \textcopyright\ 2015, Intel Corporation. All rights reserved.
\clearemptydoublepage

\pagenumbering{roman}
\tableofcontents
\clearemptydoublepage
\pagenumbering{arabic}
\hypersetup{pageanchor=true}

\chapter{Introduction}
\section{Motivation}
The versatility and ease of use of \ixp\ coprocessor family continues to underline the importance and relevance of the offload programming paradigm.
This paradigm pairs very well with another programming model, a special form of task-based parallelism, namely the \emph{stream processing} pattern.

Although users of the \ixp\ could write streaming-based applications using the \coi\ bundled with \mpssFullName, the applications were required to implement a considerable amount of plumbing.

\heterostreams\ (\hstreams\ for short) aims to relieve the developers of the burden of implementing the necessary infrastructure for applications wishing to leverage the streaming programming paradigm on heterogeneous platforms themselves by exposing a concise programming model through a well-defined API available in form of a shared library.

\chapter{Programming Overview}
\section{Source and sink}
\hstreams\ uses two special terms, \emph{source} and \emph{sink} to mark the difference between the place where the work is produced, i.e. the source and the place where the work is executed, i.e. the sink.
For example, in a computer with an \ixp coprocessor attached to its PCIe bus a source would be a process running on the host CPU while a sink would be located on the coprocessor. 
\begin{figure}[h]
    \centering
    \includegraphics[width=0.8\textwidth]{\imgdir host-and-phis.pdf}
    \caption{An illustration of the \emph{source} and \emph{sink} concepts with a host and three \ixp\ coprocessors.}
    \label{fig:host-and-phis}
\end{figure}

\section{API levels}
The \hstreams\ library exposes two levels of API - the \emph{app} API and the \emph{core} API.
app API is designed to allow a novice user to quickly start writing programs using the library and offers only a subset of the full functionality of the \hstreams\ library.
Moreover, it contains several helper functions, common building blocks which further help boosting the productivity of a beginning user.
The core API -- on the other hand -- exposes the full functionality available in the \hstreams\ library and is targeted at a more advanced user.

It is worth noting that both of those API levels are interoperable -- entities created through the app API can be manipulated and used by the core API and \emph{vice-versa}.

\section{A getting started example}\label{sec:getting-started-example}
Without getting into too much detail about the library, let us now consider a minimal working example for successfully running a compute command on an \ixp, making the use of \hstreams\ through its app API.
Listing \ref{lst:example-src.cpp} shows an example source-side application written in C++.
Listing \ref{lst:example-sink.cpp} on the other hand shows the corresponding sink-side library which contains a user-defined function -- \texttt{hello\_world} -- which is to be invoked.

\noindent\begin{minipage}{.45\textwidth}
\begin{lstlisting}[style=CppListingStyle,caption=example\_src.cpp,label={lst:example-src.cpp}]
// Main header for app API (source)
#include <hStreams_app_api.h>

int main() {
    uint64_t arg = 3735928559;
    // Create domains and streams
    hStreams_app_init(1,1);
    // Enqueue a computation in stream 0
    hStreams_app_invoke(0, "hello_world",
        1, 0, &arg, NULL, NULL, 0);
    // Finalize the library. Implicitly
    // waits for the completion of
    // enqueued actions
    hStreams_app_fini();
    return 0;
}
\end{lstlisting}
\end{minipage}\hfill
\begin{minipage}{.45\textwidth}
\begin{lstlisting}[style=CppListingStyle,caption=example\_sink.cpp,label={lst:example-sink.cpp}]
// Main header for sink API
#include <hStreams_sink.h>
// for printf()
#include <stdio.h>

// Ensure proper name mangling and symbol
// visibility of the user function to be
// invoked on the sink.
HSTREAMS_EXPORT
void hello_world(uint64_t arg)
{
    // This printf will be visible
    // on the host. arg will have
    // the value assigned on the source
    printf("Hello world, %x\n", arg);
}
\end{lstlisting}
\end{minipage}

In order to run the example, one must set the environment variable \texttt{SINK\_LD\_LIBRARY\_PATH} so that \hstreams\ runtime can pick up the appropriate libraries.
Listing \ref{lst:running-example-linux-knc} and \ref{lst:running-example-linux-x200} shows a record of a session on the Linux operating system in which both the source and the sink components of the example are compiled, the environment variable is set and finally, the application is executed.
Listing \ref{lst:running-example-windows} shows a corresponding session on the Windows operating system.

\begin{lstlisting}[style=BashCmdListingStyle,caption={Compiling and running the example on Linux on KNC card},frame=tlrb,label={lst:running-example-linux-knc}]
$ # setup Intel compiler variables
$ (...)compilervars.sh intel64
$ icc example_src.cpp -lhstreams_source -o example
$ icc -fPIC -mmic -shared example_sink.cpp -o example_mic.so
$ export SINK_LD_LIBRARY_PATH=/opt/mpss/3.7/sysroots/k1om-mpss-linux/usr/lib64/:\
> $MIC_LD_LIBRARY_PATH:\
> $(pwd)
$ ./example
Hello world, deadbeef
\end{lstlisting}

\begin{lstlisting}[style=BashCmdListingStyle,caption={Compiling and running the example on Linux on x200 card},frame=tlrb,label={lst:running-example-linux-x200}]
$ # setup Intel compiler variables
$ (...)compilervars.sh intel64
$ icc example_src.cpp -lhstreams_source -o example
$ icc -fPIC -xMIC-AVX512 -shared example_sink.cpp -o example_mic.so
$ export SINK_LD_LIBRARY_PATH=$LD_LIBRARY_PATH:$(pwd)
$ ./example
Hello world, deadbeef
\end{lstlisting}


\begin{lstlisting}[style=BatchListingStyle,caption={Compiling and running the example on Windows},frame=tlrb,label={lst:running-example-windows}]
> rem setup Intel compiler variables
> (...) compilervars.bat intel64
> icl /Qmic -fPIC -shared -o example_mic.so example_sink.cpp
> icl /I"%INTEL_MPSS_HOST_SDK%\include" example_src.cpp /Feexample.exe /link /libpath:"%INTEL_MPSS_HOST_SDK%\lib" hstreams_source.lib
> set SINK_LD_LIBRARY_PATH=%cd%;%INTEL_MPSS_HOST_SDK%\..\k1om-mpss-linux\usr\lib64;%MIC_LD_LIBRARY_PATH%;
> example.exe
Hello world, deadbeef
\end{lstlisting}

\section{Logging and traces in hetero-streams}
To aid the development and debugging process of programs which leverage hetero-streams, the library can emit detailed debug output thanks to its internal logging mechanisms.
There are two dimensions in which the output of the library can be controlled:
\begin{itemize}
    \item verbosity
    \item category of the information
\end{itemize}

The \emph{verbosity} dimension is controlled through the \texttt{hStreams\_Cfg\_SetLogLevel} API.
Using this function, user can instruct hetero-streams to never emit any logs or produce messages \emph{up to} a specific level (refer to \texttt{HSTR\_LOG\_LEVEL\_VALUES}, e.g. \texttt{HSTR\_LOG\_LEVEL\_NO\_LOGGING} or \texttt{HSTR\_LOG\_LEVEL\_DEBUG4}).

The other mechanism that hetero-streams employs to further aid the debugging is the \emph{categorisation} of messages which it produces.
And so, each message which is produced ongs to a specific category, e.g. synchronization events are described by \texttt{HSTR\_INFO\_TYPE\_SYNC} and messages related to memory actions (allocations, deallocations, \ldots) are handled by \texttt{HSTR\_INFO\_TYPE\_MEM}.

Whether messages belonging to a particular category are printed or not is controlled by a \emph{message filter} which can be adjusted by use of the \texttt{hStreams\_Cfg\_SetLogLevel} function.
This function accepts a bitmask that determines which messages are to be shown and which are to be dropped.
The values for modifying this bitmask are described by the values of the \texttt{HSTR\_INFO\_TYPE\_VALUES} enumerated type.

Single log line is formatted as follows:

\emph{Timestamp} \emph{PhysicalDomain} \emph{ProcessId} \emph{ThreadId} \emph{LogLevel} \emph{InfoType} \emph{LogMsg} (\emph{FileName}:\emph{Line})

\iffalse
Uncomment if needed

\begin{itemize}
    \item \emph{Timestamp} is number of milliseconds from linux epoch.
    \item \emph{PhysicalDomain} is number of physical domain.
    \item \emph{ProcessId} is id of process which made call to hStreams function.
    \item \emph{ThreadId} is id of thread which made call to hStreams function.
    \item \emph{LogLevel} is one of \texttt{HSTR\_LOG\_LEVEL\_VALUES}
    \item \emph{InfoType} is one of \texttt{HSTR\_INFO\_TYPE\_VALUES}
    \item \emph{LogMsg} is log message.
    \item \emph{FileName} is name of file from which log message comes from.
    \item \emph{Line} is line number in \emph{FileName} from which log message comes from.
\end{itemize}
\fi

\section{Infrastructure}
\subsection{\hstreams\ files}
\hstreams\ exposes its functionality through dynamic shared objects that user applications can load and then call functions from that object. The binary artifacts provided with \hstreams\ are shown in Table \ref{tab:hstreams-binaries-linux} for \linux\ operating system and in Table \ref{tab:hstreams-binaries-windows} for \windows\ operating system.
\begin{table}[h]
\begin{tabularx}{\textwidth}{ l X }\hline
  Binary file name & Description \\ \hline
  \texttt{libhstreams\_source.so} & A dynamic shared library providing ,,source'' functionality \\ \hline
\end{tabularx}
\caption{Binary artifacts of the \hstreams\ library on \linux\ operating system.}
\label{tab:hstreams-binaries-linux}
\end{table}

\begin{table}[h]
\begin{tabularx}{\textwidth}{ l X }\hline
  Binary file name & Description \\ \hline
  \texttt{hstreams\_source.dll} & A dynamic-link library providing ,,source'' functionality \\ \hline
\end{tabularx}
\caption{Binary artifacts of the \hstreams\ library on \windows\ operating system.}
\label{tab:hstreams-binaries-windows}
\end{table}

The function entry points and types are declared in the header files listed in Table \ref{tab:hstreams-headers}.
\begin{table}[h]
\begin{tabularx}{\textwidth}{ l X }\hline
  Header file name & Description \\ \hline
  \texttt{hStreams\_app\_api.h} & Declarations for source-side app API \\ \hline
  \texttt{hStreams\_common.h} & Declarations which are common to both sink and source \\ \hline
  \texttt{hStreams\_sink.h} & Sink-side declarations, e.g. \texttt{HSTREAMS\_EXPORT} \\ \hline
  \texttt{hStreams\_source.h} & Declarations for source-side core API \\ \hline
  \texttt{hStreams\_types.h} & Declarations of types used in APIs, both source and sink \\ \hline
  \texttt{hStreams\_version.h} & Preprocessor definitions for the \hstreams\ library version \\ \hline
\end{tabularx}
\caption{Header files of the \hstreams\ library.}
\label{tab:hstreams-headers}
\end{table}

\section{Common issues with setting up the environment}
This section attempts to capture most common issues users may encounter with an improper setup of the environment in which they attempt to execute a binary making use of \hstreams.
To illustrate the possible failures, the example from Section \ref{sec:getting-started-example} will be used.

If the \texttt{SINK\_LD\_LIBRARY\_PATH} variable is not set at all, the following error will be presented:
\begin{lstlisting}[style=BashCmdListingStyle,caption={Running the example without \texttt{SINK\_LD\_LIBRARY\_PATH}},frame=tlrb,label={lst:example-nosinkld}]
$ unset SINK_LD_LIBRARY_PATH
$ ./example
hStreams_Init_worker: returns: HSTR_RESULT_DEVICE_NOT_INITIALIZED, Did not find libhstreams_mic on host.
\end{lstlisting}

In the next example, the \texttt{SINK\_LD\_LIBRARY\_PATH} variable contains only the location of the \texttt{libhstreams\_mic} binary.
\begin{lstlisting}[style=BashCmdListingStyle,caption={Running the example with incomplete \texttt{SINK\_LD\_LIBRARY\_PATH}},frame=tlrb,label={lst:example-incompletesinkld0}]
$ export SINK_LD_LIBRARY_PATH=/opt/mpss/3.7/sysroots/k1om-mpss-linux/usr/lib64/
$ ./example
The remote process indicated that the following libraries could not be loaded:  libmkl_intel_lp64.so libmkl_intel_thread.so libmkl_core.so libiomp5.so libimf.so libsvml.so libirng.so libintlc.so.5
hStreams_Init_worker: returns: HSTR_RESULT_REMOTE_ERROR, Could not create process on the device: COI_MISSING_DEPENDENCY.
\end{lstlisting}

Finally, let us see an example in which all \hstreams-required paths are set properly but the library cannot find user's custom kernels library
\begin{lstlisting}[style=BashCmdListingStyle,caption={Running the example with user's custom kernels library location missing from \texttt{SINK\_LD\_LIBRARY\_PATH}},frame=tlrb,label={lst:example-incompletesinkld1}]
$ export SINK_LD_LIBRARY_PATH=/opt/mpss/3.7/sysroots/k1om-mpss-linux/usr/lib64/:$MKLROOT/lib/mic/:$MKLROOT/../compiler/lib/mic
$ ./example
enqueueFunction: returns: HSTR_RESULT_BAD_NAME, hStreams_fetchSinkFuncAddress: returns: HSTR_RESULT_BAD_NAME, dlsym() returned error: /tmp/coi_procs/1/99657/libhstreams_mic: undefined symbol: hello_world, for function: hello_world.
A sink-compiled version of called function hello_world not found on logical domain 1, physical domain 0.
hStreams_app_invoke2: returns: HSTR_RESULT_BAD_NAME, hStreams_EnqueueCompute( in_LogStreamID, in_pFuncName, in_NumScalarArgs, in_NumHeapArgs, in_pArgs, out_pEvent, out_pReturnValue, in_ReturnValueSize) returned HSTR_RESULT_BAD_NAME
\end{lstlisting}

\section{Execution Model}
\subsection{Streams}
The basic building block of the execution model is a \emph stream which has two endpoints -- one on the \emph{source} (where actions are placed into the stream and another one on the \emph{sink}) where the actions are executed.
The sink endpoint of a stream is defined by specifying a subset of available computing resources (hardware threads) on which the actions may execute.
As they are processed, the actions to be executed resemble a \emph{first-in first-out} queue on the sink with the source endpoint of the stream pushing the actions into the queue and the sink endpoint popping them out.

All the actions which can be placed into a stream fall into one of three categories:
\begin{itemize}
    \item Compute actions
    \item Memory movement actions
    \item Wait/synchronization actions
\end{itemize}

\emph{Compute} actions are executions of functions queued up in a stream.
A compute action consists of the name of the function to be invoked, the parameters to be copied by value for remote invocation, the parameters which belong to pre-registered buffers and the output parameters of the function.

\emph{Memory movement} actions are transfers of memory contents of pre-registered buffers between arbitrary endpoints of any two streams.

\emph{Wait and synchronization} actions involve the sink endpoint of a stream waiting on a collection of events.

\subsection{Logical and physical entities}
\subsubsection{Domains and streams}
\hstreams\ introduces an abstraction of the physical resources through the existence of \emph{logical} domains, streams and buffers.
To explain the purpose and nature of those, one must introduce a few concepts - \emph{physical} streams, domains and buffers.

A \emph{physical domain} is a representation of all resources within one memory coherence domain - memory and processors.
\emph{Physical streams} are entities consisting of threads in the \emph{physical domain} bounded by a prescribed affinity mask restricted to fully contained inside the domain's processor set.

A \emph{logical domain} is a subset of a given physical domain, defined by its processor mask which must be contained within the owning physical domain's processor set.
One or more logical domains may belong to the same physical domain.
Additionally, two logical domains are prohibited from partially overlapping.
Two logical domains can, however, fully overlap (i.e. be described by identical processor masks), in which case they are considered to be distinct, unrelated entities.

Next, a \emph{logical stream} is an abstraction of a stream which resides in one given logical domain.
As with domains, a logical stream is defined by its hardware thread mask.
Multiple logical streams are allowed to be partially or fully overlapping.
Moreover, logical streams described by identical processor masks are considered to be aliasing the same underlying physical stream.
There is no correspondence assumed between logical streams with partially overlapping CPU masks, each of them is mapped to a different physical stream.

Physical streams are entities which are not enumerable via the interface of the \hstreams\ library.
They are implicitly created and destroyed when logical streams are created and destroyed.

To better understand these concepts, please refer to Figure \ref{fig:phys-log-dom-str}.
\begin{figure}[h]
    \centering
    \includegraphics[width=0.8\textwidth]{\imgdir domains-and-streams.pdf}
    \caption{An example arrangement of physical domains, logical domains, physical streams and logical streams.}
    \label{fig:phys-log-dom-str}
\end{figure}
In this picture you can see how there are three logical domains on physical domain \#0 with 21 streams spread across them.
In logical domain \#2 there are six logical streams constructed with the special property that e.g. streams \#3, \#12 and \#21 have been constructed with exactly overlapping CPU masks and thus alias the same \emph{physical} stream.
The physical streams are not given any IDs in the picture to underline the fact that they are not enumerable through the interface of the \hstreams\ library.

\subsubsection{Buffers}
In order to operate on memory resources (e.g. transfer data to and from remote domains or to supply a remote memory address as an operand of a compute action), users must make \hstreams\ aware of such memory resources.
In order to do that, user must create a \emph{logical buffer} through a call to the appropriate API supplying the start address and the length of that buffer.
Once the \hstreams\ runtime is aware of the buffer, a pointer to a memory location anywhere \emph{inside} of that buffer is recognized as a handle and can be used for performing data transfers or compute actions involving that buffer.

A logical buffer created by the user may have instantiations in many logical domains beside the source.
Those instantiations of the buffer are called \emph{physical buffers}.
A logical buffer must have a corresponding physical buffer on the logical domain where it is intended to be used (either as an operand of a data transfer or a compute action).
E.g. if a buffer is supplied as an argument to a compute action placed into a stream the sink endpoint of which is located in logical domain \#4, then the logical buffer must be already \emph{instantiated} in logical domain \#4.

Please note that the contents of multiple buffer instantiations are not synchronized automatically, i.e. if a buffer is instantiated for a logical domain \#12, the contents of that instantiation are undefined until they are determined explicitly by the user either writing to the instantiation from a compute action or transferring data to that instantiation.

